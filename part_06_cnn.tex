% !TEX root = main.tex

%%%-------------------------------------------
\section{Свёрточные сети}

\epigraph{Какая-нибудь цитата про сворачиваемся}{автор цитаты}

% Нарезать на задачи: 
% https://arxiv.org/pdf/1603.07285.pdf

\begin{problem}{(Свёртка своими руками)}
Дать картинку и попросить сделать свертку с падингом/без/страйдами и тп
\end{problem}


\begin{problem}{(Ядра)}
У Маши есть куча ядер для свёрток. Догадайтесь какое из них что делает: 

- границы
- повороты
- сдвиги 
\end{problem}


\begin{problem}{(Крестики, нолики и слэши)}
На лекции по нейросеткам Маша увидела ядро для поиска слэшей. 

\todo[inline]{Пример ядра}

Маша хочет придумать похожее ядро для классификации крестиков и ноликов. Сюда примеры ноликов и крестиков в размерностях 5 на 5.
\end{problem}



\begin{problem}{(Свёрточный и полносвязный)}
Маше рассказали, что свёрточный слой --- это полносвязный слой с некоторыми ограничениями. Она хочет разобраться что это за ограничения. Помогите ей записать свёрточный слой в виде полносвязного и разобраться. 

На вход в слой идёт чёрно-белое изображение. Нарисуйте картинку с полносвязным слоем и свёрточным в виде полносвязного. Подпишите все веса. Запишите свёрточный слой с помощью перемножения матриц в виде $H = X \cdot W.$ Как выглядит матрица $W$? 
\end{problem}


\begin{problem}{(Число параметров)}
На вход в нейронную сетку идёт изображение размера $28 \times 28$.

\begin{enumerate}
\item Маша вытягивает эту картинку в длинный вектор и использует полносвязную сетку для классификации изображений. В сетке идёт один полносвязный слой из $1000$ нейронов. После идёт слой, который осуществляет классификацию изображения на $10$ классов. Сколько параметров нужно оценить?

\item \todo[inline]{То же самое но с подсчётом числа параметров в LaNet}

\end{enumerate}
\end{problem}
\begin{sol}
У нас $28^2 = 784$ входа. Весов между входным и полносвязным слоями будет 

\[ (784 + 1)\cdot 1000 = 785000.\] 

Единица отвечает за константу для каждого из $1000$ нейронов. Между полносвязным и итоговым слоем

\[(1000 + 1) \cdot 10 = 10010. \]
\end{sol} 

\begin{problem}{(Число параметров [2])}

Маша собирает разные архитектуры. Помогите ей оценить число параметров для каждой из них. 

\begin{enumerate} 
    \item У Маши есть свёрточный слой. На вход в свёрточный слой идёт изображение с $C_{in}$ каналами размера $W \times H$. Маша использует $C_{out}$ фильтров размера $W_k \cdot H_k$. Сколько параметров ей предстоит оценить?
    
    \item 
    
    \item  \todo[inline]{Сюда вариант с подсчётом числа параметров в сепарабельной свёртке}
    
\end{enumerate} 
\end{problem}
\begin{sol}
$(W_k \cdot H_k \cdot C_{in} + 1) \cdot C_{out}$
\end{sol} 

\begin{problem}{(Поле восприятия)}
    Маша хочет найти котика размера $512 \times 512$ пикселей. Для этого она использует свёртки размера $5 \times 5$ без дополнения нулями (padding).  с пропусками (strides). После каждого свёрточного слоя Маша делает пулинг. 
    
    Через сколько слоёв поле восприятия Машиной нейросетки впервые охватит котейку. 
    % Нарезать на задачи: https://distill.pub/2019/computing-receptive-fields/
\end{problem}



